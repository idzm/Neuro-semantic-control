\chapter*{Введение}
\addcontentsline{toc}{chapter}{Введение}

В современных условиях управление производством становится все сложнее, требования к эффективности более высокими. Задачи, находящиеся на уровне АСУТП, находятся в тесном взаимодействии как с верхним уровнем (планирование производства и т.п.), так и с нижним (уровень технологического оборудования). Один из путей улучшения производства может заключаться за счет совершенствования применяемых на уровне АСУТП подходов к управлению – применению последних разработок в данной области, одной из которых является нейроуправление.

ПИ- и ПИД-регуляторы были одними из первых систем управления \cite{Omatu_Khalid_Yusof}. Они зарекомендовали себя как относительно простые и надежные системы, которые достаточно эффективно решали поставленные задачи. И в настоящее время они остаются преобладающими системами управления, несмотря на наличие в них определенных недостатков и ограничений, которых лишено нейроуправление. Новые подходы позволяют строить более эффективные системы управления по сравнению с классическими ПИД-регуляторами.

Нейроуправление – относительно молодое направление научных исследований, которое стало самостоятельным в 1988 году. Однако исследования в этой области начались гораздо раньше. Одно из определений науки «кибернетика» рассматривает ее как общую теорию управления и взаимодействия не только машин, но и биологических существ. Нейроуправление пытается реализовать данное положение через построения систем управления (систем принятия решений), которые могут обучаться во время функционирования, и таким образом, улучшать свою эффективность работы. При этом такие системы используют параллельные механизмы обработки информации, подобно мозгу живых организмов \cite{Uskov_2004}.

Долгое время была популярна идея построения совершенной системы управления – универсального контроллера, который извне выглядел бы как «черный ящик». Он мог бы использоваться для управления любыми системами, имея связи с датчиками, исполнительными механизмами, другими контроллерами и специальную связь с «модулем эффективности» – системой, которая определяет эффективность управления исходя из заданных критериев. Пользователь такой системы управления задавал бы только желаемый результат, далее обученный контроллер управлял бы самостоятельно, возможно придерживаясь сложной стратегии достижения в будущем желаемого результата. Также он бы все время корректировал свое управление исходя из реакции объекта управления для достижения максимальной эффективности. Общая схема такой системы приведена ниже.

\begin{figure}[H]
    % Use ’standalone’ as document class:
\documentclass{standalone}
% Load packages needed for this TeX file:
\usepackage{tikz}
\usetikzlibrary{quotes,positioning,arrows.meta,arrows,calc}

\usepackage[T1,T2A]{fontenc}
\usepackage[utf8]{inputenc}

% Surround TeX code with ’document’ environment as usually:
\begin{document}
% Add your TeX code, e.g. a picture:

\tikzset{
    box/.style={
            % The shape:
            rectangle,
            % The size:
            minimum size=20mm,
            % The border:
            very thick,
            draw=red!50!black!50,         % 50% red and 50% black,
            % and that mixed with 50% white
            % The filling:
            top color=white,              % a shading that is white at the top...
            bottom color=red!50!black!20, % and something else at the bottom
            % Font
            font=\itshape,
            align=center}
}

\tikzset{big_arrow/.style={-{Stealth[length=5mm, width=4mm]}}}

\begin{tikzpicture}[
        right1/.style={to path={-- ++(5,0) |- (\tikztotarget)}},
        left1/.style={to path={-- ++(-5,0) |- (\tikztotarget)}}]
    \node (o1)   [box]                        {Объект\\управления};
    \node (u1)   [below=of o1,align=center]   {$\mathbf{ U(t) }$\\Эффективность};
    \node (c1)   [box,below=of u1]            {Управляющее\\устройство\\(контроллер)};

    \node (control) [right=of c1,align=center]  {$\mathbf{ u(t) }$\\Управление};
    \node (sensors) [left=of c1,align=center]   {$\mathbf{ X(t) }$\\Показания датчиков};

    \path {
        (o1)            edge[very thick]                     (u1)
        (u1)            edge[very thick, big_arrow]          (c1)
        ($ (c1.east) $) edge[very thick, big_arrow, right1]  ($ (o1.east) $)
        ($ (o1.west) $) edge[very thick, big_arrow, left1]   ($ (c1.west) $) };

\end{tikzpicture}
\end{document}

    \centering
    \caption{Система с подкрепляющим обучением}
    \label{fig:reinforce_learning_system}
\end{figure} 

В настоящее время не только оборудование, применяемое на ОАО «Савушкин продукт», характеризуется очень высокой сложностью, но и технологические процессы также. Настройка параметров технологических линий требует наличия специалистов высокого уровня и занимает длительное время. Соответственно требования к качеству изготовления продукции очень высоки, так как от него напрямую зависит размер получаемой прибыли (выше качество – лучше потребительские характеристики товара – дольше срок хранения – более широкие возможности по географическому охвату рынка и т.д.). Нейроуправление позволяет повысить качество продукции за счет повышения эффективности управления, а также ускорить настройку параметров. Поэтому актуальной является задача применения нейроуправления для построения сложных управляющих систем на уровне АСУТП, которые были бы лишены недостатков, присущих используемым системам (на основе ПИД-регуляторов).
