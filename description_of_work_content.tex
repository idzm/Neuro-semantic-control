\newcommand{\actuality}{\textbf{Связь работы с научными программами (проектами), темами}}
\newcommand{\aim}{\textbf{Цель, задачи, объект и предмет исследования}}
\newcommand{\novelty}{\textbf{Научная новизна}}
\newcommand{\defpositions}{\textbf{Положения, выносимые на защиту}}
\newcommand{\influence}{\textbf{Научная и практическая значимость}}
\newcommand{\reliability}{\textbf{Степень достоверности}}
\newcommand{\contribution}{\textbf{Личный вклад соискателя ученой степени в результаты диссертации}}
\newcommand{\probation}{\textbf{Апробация диссертации и информация об использовании ее результатов}}
\newcommand{\publications}{\textbf{Опубликованность результатов диссертации}}

%%http://www.linux.org.ru/forum/general/6993203#comment-6994589 (используется totcount).
\makeatletter
\def\formbytotal#1#2#3#4#5{%
    \newcount\@c
    \@c\totvalue{#1}\relax
    \newcount\@last
    \newcount\@pnul
    \@last\@c\relax
    \divide\@last 10
    \@pnul\@last\relax
    \divide\@pnul 10
    \multiply\@pnul-10
    \advance\@pnul\@last
    \multiply\@last-10
    \advance\@last\@c
    \total{#1}~#2%
    \ifnum\@pnul=1#5\else%
    \ifcase\@last#5\or#3\or#4\or#4\or#4\else#5\fi
    \fi
}
\makeatother

\input{characteristic} % Характеристика работы по структуре во введении и в автореферате не отличается (ГОСТ Р 7.0.11, пункты 5.3.1 и 9.2.1), потому её загружаем из одного и того же внешнего файла, предварительно задав форму выделения некоторым параметрам.

\vspace{3mm}
\textbf{Структура и объем диссертации}
\vspace{3mm}

Диссертация состоит из введения, общей характеристики работы, четырех глав с краткими выводами по каждой главе, заключения, библиографического списка, списка публикаций автора и приложений.

В \textbf{\textit{первой главе}} рассмотрено краткое описание современных систем управления, дана классификация их типов, определены понятия символического нейроуправления и семантических технологий, рассмотрена классификация нейросетевых приёмов управления, онтология на основе стандарта ISA-88. \textbf{\textit{Вторая глава}} посвящена рассмотрению нейросетевым моделям управления технологическими процессами, \textbf{\textit{третья глава}} - семантическим моделям управления технологическими процессами. В \textbf{\textit{четвертой главе}} приведены результаты разработки нейро-символической модели управления процессом пастеризации.

\textcolor{red}{
    Общий объём диссертации составляет \formbytotal{TotPages}{страниц}{у}{ы}{}, из которых \formbytotal{textpages}{страниц}{у}{ы}{} основного текста,
    \iftotalfigures
        \formbytotal{totalcount@figure}{рисун}{ок}{ка}{ков},
    \fi
    \iftotaltables
        \formbytotal{totalcount@table}{таблиц}{а}{ы}{},
    \fi
    библиография из \formbytotal{citenum}{источник}{}{а}{ов}, включая \formbytotal{citenum_my}{публикац}{ия}{ии}{ий} автора.}
