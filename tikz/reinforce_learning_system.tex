% Use ’standalone’ as document class:
\documentclass{standalone}
% Load packages needed for this TeX file:
\usepackage{tikz}
\usetikzlibrary{quotes,positioning,arrows.meta,arrows,calc}

\usepackage[T1,T2A]{fontenc}
\usepackage[utf8]{inputenc}

% Surround TeX code with ’document’ environment as usually:
\begin{document}
% Add your TeX code, e.g. a picture:

\tikzset{
    box/.style={
            % The shape:
            rectangle,
            % The size:
            minimum size=20mm,
            % The border:
            very thick,
            draw=red!50!black!50,         % 50% red and 50% black,
            % and that mixed with 50% white
            % The filling:
            top color=white,              % a shading that is white at the top...
            bottom color=red!50!black!20, % and something else at the bottom
            % Font
            font=\itshape,
            align=center}
}

\tikzset{big_arrow/.style={-{Stealth[length=5mm, width=4mm]}}}

\begin{tikzpicture}[
        right1/.style={to path={-- ++(5,0) |- (\tikztotarget)}},
        left1/.style={to path={-- ++(-5,0) |- (\tikztotarget)}}]
    \node (o1)   [box]                        {Объект\\управления};
    \node (u1)   [below=of o1,align=center]   {$\mathbf{ U(t) }$\\Эффективность};
    \node (c1)   [box,below=of u1]            {Управляющее\\устройство\\(контроллер)};

    \node (control) [right=of c1,align=center]  {$\mathbf{ u(t) }$\\Управление};
    \node (sensors) [left=of c1,align=center]   {$\mathbf{ X(t) }$\\Показания датчиков};

    \path {
        (o1)            edge[very thick]                     (u1)
        (u1)            edge[very thick, big_arrow]          (c1)
        ($ (c1.east) $) edge[very thick, big_arrow, right1]  ($ (o1.east) $)
        ($ (o1.west) $) edge[very thick, big_arrow, left1]   ($ (c1.west) $) };

\end{tikzpicture}
\end{document}
